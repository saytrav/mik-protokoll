\section{Auswertung}

\subsection{Experiment 1}

\subsection{Experiment 2}

\subsection{Experiment 3}

First the resolution limit has to be measured:

\begin{align*}
    B_{1} & = 0,3 \cdot 10^-3 m \\
    B_{2} & = 0,6 \cdot 10^-3 m \\
    B &= (0,45 \pm 0,15) \cdot 10^-3 m
\end{align*}

Using the eqation $\ref{7}$ $\epsilon$ is calculated:

\begin{align*}
    tan(\epsilon) & = 4,5 \cdot 10^-3 \\
    \sigma tan(\epsilon) &= (\sigma B)/2 + \sigma f_{ob} \\
    &= 0,19 \\
    \Delta tan(\epsilon) &= 8,55 \cdot 10^-4  \\
    tan(\epsilon) & = (4,5 \cdot 10^-3 \pm 8,55 \cdot 10^-4) \\
    \epsilon &= 0,26 \circ \\
    \Delta \epsilon + \epsilon &= arctan(4,5 \cdot 10^-3 + 8,55 \cdot 10^-4) \\
    &= 0,31 \circ \\
    \Delta \epsilon &= 0,05 \circ \\
    \epsilon &= (0,26 \pm 0,05) \circ
\end{align*}

Since the diffraction index n for air is equal to one, the numercal aperture is equal to sin($\epsilon$) (see $\ref{8.1}$):

\begin{align*}
  A & = 4,54 \cdot 10^-3 \\
  \Delta A + A & = sin(\epsilon + \Delta \epsilon) \\
  &= 5,41 \cdot 10^-3 \\
  \Delta A &= 8,72 \cdot 10^-4 \\
  A & = (4,54 \pm 0,87 ) \cdot 10^-3
\end{align*}

According to $\ref{8}$, knowing that the wavelength $\lambda$ is equal to 1, all the necessary values to calculate 
$d_{min}$ are availalble:

\begin{align*}
  d_{min} & = 1,21 \cdot 10^-4 m \\
  d_{min} + \Delta d_{min} & = \frac{550 \cdot 10^-9 m}{sin(\epsilon - \Delta \epsilon)} \\
  \Delta d_{min} &= 2,9 \cdot 10^-4 m \\
  d_{min} & = (1,21 \pm 2,9) \cdot 10^-4
\end{align*}

$d_{min}$ is a smaller value than the grid constant d.

  \newpage
